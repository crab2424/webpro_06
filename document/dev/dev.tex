\documentclass[a4paper,12pt,oneside]{ltjsarticle}
\usepackage{lmodern}
\usepackage{listings} % コード表示用
\usepackage{longtable} % ページをまたぐ表用
\usepackage{color}
\usepackage{graphicx}% 画像読み込み用
\usepackage{xcolor}
\usepackage[cmex10]{amsmath}
\usepackage[hyphens]{url}
\usepackage{float}
\usepackage{moreverb}
\usepackage{lscape}
\usepackage{cite}
\usepackage{ascmac}
\usepackage{amssymb}
\usepackage{colortbl}
\usepackage{float}
\usepackage{textcomp}
\usepackage{multirow,multicol}
\usepackage{wrapfig}
\usepackage{minted}


\lstdefinelanguage{JavaScript}{
  keywords={typeof, new, true, false, catch, function, return, null, catch,
    switch, var, let, const, if, while, do, else, case, break},
  keywordstyle=\color{blue}\bfseries,
  ndkeywords={class, export, boolean, throw, implements, import, this},
  ndkeywordstyle=\color{teal}\bfseries,
  identifierstyle=\color{black},
  sensitive=false,
  comment=[l]{//},
  morecomment=[s]{/*}{*/},
  commentstyle=\color{gray}\ttfamily,
  stringstyle=\color{orange}\ttfamily,
  morestring=[b]',
  morestring=[b]"
}

% --- コードハイライト設定 ---
\definecolor{commentgreen}{rgb}{0,0.6,0}
\definecolor{stringpurple}{rgb}{0.58,0,0.82}
\definecolor{backcolour}{rgb}{0.95,0.95,0.92}

\lstset{
    backgroundcolor=\color{backcolour},
    commentstyle=\color{commentgreen},
    keywordstyle=\color{blue},
    stringstyle=\color{stringpurple},
    basicstyle=\ttfamily\footnotesize,
    breakatwhitespace=false,
    breaklines=true,
    captionpos=b,
    keepspaces=true,
    numbers=left,
    numbersep=5pt,
    showspaces=false,
    showstringspaces=false,
    showtabs=false,
    tabsize=2,
    frame=single,
    language=JavaScript
}



% --- タイトル情報 ---
\title{開発者向け仕様書}
\author{開発チーム}
\date{\today}

\begin{document}

\maketitle
\tableofcontents
\newpage

\section{システム概要}
本システムは、Node.js (Express) を用いたサーバーサイドレンダリング (SSR) Webアプリケーションである。
3つの異なるデータセット(都道府県、元素記号、88星座)に対し、RESTfulな設計に基づいたCRUD(作成、読み取り、更新、削除)機能を提供する。

\subsection{特徴と制約}
\begin{itemize}
    \item \textbf{データ永続化なし}: データベースを使用せず、\textbf{サーバーサイドの変数(メモリ上の配列)}にデータを保持する。サーバーを再起動すると、データは初期状態にリセットされる。
    \item \textbf{テンプレートエンジン}: EJSを使用し、HTMLを動的に生成する。
    \item \textbf{拡張性}: ルーティングとビューを追加することで、容易に新しいテーマ(例:商品管理、書籍リストなど)を追加可能である。
\end{itemize}

\section{アーキテクチャ構成}
システムはMVC(Model-View-Controller)モデルに近い構成をとるが、簡易化のためModel部分は変数操作として実装する。

\subsection{ディレクトリ構成案}
開発者が直感的にファイルを配置できるよう、以下の構成を推奨する。

\begin{lstlisting}[caption=ディレクトリ構成, language=bash]
root/
|-- app.js                # エントリーポイント
|-- routes/               # ルーティング定義 (Controller相当)
|   |-- prefectures.js    # 都道府県用ルーター
|   |-- elements.js       # 元素記号用ルーター
|   `-- constellations.js # 88星座用ルーター
|-- views/                # EJSテンプレート (View)
|   |-- index.ejs         # トップページ
|   |-- prefectures/      # 都道府県用ビューフォルダ
|   |   |-- index.ejs     # 一覧表示
|   |   |-- show.ejs      # 詳細表示
|   |   |-- new.ejs       # 新規作成フォーム
|   |   `-- edit.ejs      # 編集フォーム
|   |-- elements/         # (構成は同上)
|   `-- constellations/   # (構成は同上)
|-- public/               # 静的ファイル (css, images)
`-- package.json          # 依存関係定義
\end{lstlisting}

\section{データモデル仕様 (Model)}
データは各ルーターファイル(例: \texttt{routes/prefectures.js})内のローカル変数として定義する。

\subsection{変数定義のルール}
\begin{itemize}
    \item \textbf{型}: オブジェクトの配列 (\texttt{Array<Object>})
    \item \textbf{初期データ}: サーバー起動時に配列に格納されるデフォルトデータ。
    \item \textbf{ID管理}: 新規追加時は、既存の最大ID + 1 を付与するか、単純なインクリメントロジックを実装する。
\end{itemize}

\begin{lstlisting}[caption=routes/prefectures.js の実装例]
// データストア(変数)
let prefectures = [
    { id: 1, name: '北海道', region: '北海道', capital: '札幌市' },
    { id: 2, name: '青森県', region: '東北', capital: '青森市' },
    // ...初期データ
];

// 次のIDを管理する変数
let nextId = 48;
\end{lstlisting}

\section{APIエンドポイントと処理フロー}

\subsection{実装すべきルート処理詳細}
各テーマ共通で実装するURLパターンと処理内容は以下の通りである。

\begin{longtable}{|l|c|p{8cm}|}
\hline
\textbf{URLパターン} & \textbf{メソッド} & \textbf{処理概要と注意点} \\
\hline
\endhead

\texttt{/} & GET & \textbf{一覧表示} \newline 配列全体を \texttt{views/index.ejs} に渡す。 \\
\hline
\texttt{/new} & GET & \textbf{新規作成画面} \newline 空のフォームを表示する。 \\
\hline
\texttt{/} & POST & \textbf{データ作成} \newline \texttt{req.body}から値を取得し、IDを付与して配列に\texttt{push}する。その後一覧へリダイレクト。 \\
\hline
\texttt{/:id} & GET & \textbf{詳細表示} \newline \texttt{req.params.id}と一致する要素を配列から\texttt{find}で検索し表示。見つからない場合は404。 \\
\hline
\texttt{/:id/edit} & GET & \textbf{編集画面} \newline 対象データを検索し、フォームの初期値(\texttt{value})として渡して表示。 \\
\hline
\texttt{/:id} & POST & \textbf{データ更新} \newline 対象データを検索し、\texttt{req.body}の内容でプロパティを上書きする。 \\
\hline
\texttt{/:id/delete} & POST & \textbf{データ削除} \newline \texttt{req.params.id}と一致する要素を配列から除外(\texttt{filter}等)する。 \\
\hline
\end{longtable}

\section{拡張・変更ガイド}

\subsection{ケースA:データの項目を増やしたい}
例:都道府県に「名産品 (\texttt{product})」を追加する場合。

\begin{enumerate}
    \item \textbf{データの変更}: \texttt{routes/prefectures.js} の初期データ配列に \texttt{product} プロパティを追加する。
    \item \textbf{新規作成処理の変更}: 同ファイルの \texttt{POST /} ルート内で、\texttt{req.body.product} を受け取るように修正する。
    \item \textbf{更新処理の変更}: 同ファイルの \texttt{POST /:id} ルート内で、更新ロジックに \texttt{product} を追加する。
    \item \textbf{ビューの変更}:
    \begin{itemize}
        \item \texttt{index.ejs}: 表に見出しと列を追加。
        \item \texttt{show.ejs}: 表示項目を追加。
        \item \texttt{new.ejs} / \texttt{edit.ejs}: \texttt{<input name="product">} タグを追加。
    \end{itemize}
\end{enumerate}

\subsection{ケースB:新しいテーマ(例:戦国武将)を追加したい}
\begin{enumerate}
    \item \textbf{ルーター作成}: \texttt{routes/warriors.js} を作成し、CRUDルートとデータ配列(\texttt{warriors})を記述する。
    \item \textbf{ビュー作成}: \texttt{views/warriors/} フォルダを作成し、4つのejsファイルを作成する。
    \item \textbf{アプリ登録}: \texttt{app.js} に以下を追記する。
\begin{lstlisting}
const warriorsRouter = require('./routes/warriors');
app.use('/warriors', warriorsRouter);
\end{lstlisting}
\end{enumerate}

\section{技術スタック要件}
\begin{itemize}
    \item \textbf{Runtime}: Node.js (v14以上推奨)
    \item \textbf{Framework}: Express
    \item \textbf{Template Engine}: EJS
    \item \textbf{Body Parser}: Express標準 (\texttt{express.urlencoded}) ※POSTデータ受け取りに必須
\end{itemize}

\end{document}
