\chapter{開発者向け仕様書:都道府県一覧表示システム}

\section{概要}
本ドキュメントは,Node.jsおよびテンプレートエンジンEJSを用いた「都道府県管理Webアプリケーション」の設計仕様書である.
本システムは,サーバーサイドで都道府県データを管理し,EJSを用いて動的にHTMLを生成・返却する.
データベースの利用は行わず,サーバープロセスのメモリ上(変数)でデータを保持・操作することを前提とする.

\section{システム要件}

\subsection{技術スタック}
\begin{itemize}
    \item \textbf{ランタイム}: Node.js
    \item \textbf{Webフレームワーク}: Express
    \item \textbf{テンプレートエンジン}: EJS
    \item \textbf{データ保存先}: サーバーサイドメモリ(配列変数)
\end{itemize}

\subsection{データ構造 (Schema)}
サーバー内の配列変数で管理するオブジェクト構造は以下の通り.

\begin{table}[H]
    \centering
    \caption{都道府県データスキーマ}
    \label{tab:schema}
    \begin{tabular}{llp{8cm}}
        \toprule
        プロパティ名 & データ型 & 説明 \\
        \midrule
        \texttt{id} & Number & 一意な識別子.登録時に自動採番(最大値+1 等). \\
        \texttt{name} & String & 都道府県名(例: 大阪府). \\
        \texttt{capital} & String & 県庁所在地(例: 大阪市). \\
        \texttt{region} & String & 地方区分(例: 近畿). \\
        \bottomrule
    \end{tabular}
\end{table}

\section{ディレクトリ・ページ構成}

Expressの標準的な構成に基づき,ビュー(EJS)ファイルを配置する.

\begin{verbatim}
project-root/
    |- app.js           (メインロジック・データ変数保持)
    |- public/          (CSS, 画像などの静的ファイル)
    |- views/           (EJSテンプレート)
        |- index.ejs    (一覧表示画面)
        |- show.ejs     (詳細表示画面)
        |- new.ejs      (新規登録フォーム)
        |- edit.ejs     (編集フォーム)
\end{verbatim}

\section{HTTPメソッドとルーティング}
RESTfulな設計に基づき,各URLとHTTPメソッド,および対応する処理を定義する.
※ブラウザのHTMLフォーム標準仕様ではPUT/DELETEが使用できないため,\texttt{method-override}ライブラリを使用するか,POSTメソッドで代用する想定とする(本仕様では論理的なメソッドを記載).

\begin{table}[H]
    \centering
    \caption{ルーティング一覧}
    \label{tab:routing}
    \begin{tabular}{llll}
        \toprule
        機能 & メソッド & パス (URL) & 対応ビュー/処理 \\
        \midrule
        一覧表示 & GET & \texttt{/prefectures} & \texttt{views/index.ejs} \\
        新規作成フォーム & GET & \texttt{/prefectures/new} & \texttt{views/new.ejs} \\
        詳細表示 & GET & \texttt{/prefectures/:id} & \texttt{views/show.ejs} \\
        編集フォーム & GET & \texttt{/prefectures/:id/edit} & \texttt{views/edit.ejs} \\
        \midrule
        新規データ作成 & POST & \texttt{/prefectures} & (処理後一覧へリダイレクト) \\
        データ更新 & PUT & \texttt{/prefectures/:id} & (処理後詳細へリダイレクト) \\
        データ削除 & DELETE & \texttt{/prefectures/:id} & (処理後一覧へリダイレクト) \\
        \bottomrule
    \end{tabular}
\end{table}

\section{ページ遷移図}

画面間の遷移フローを以下に示す.

\vspace{1cm}
% フローチャート配置場所
\begin{figure}[H]
    \centering
    \includegraphics[width=\textwidth]{fig/dev_prefecture/flowchart-1.pdf}
    \caption{画面遷移フロー}
    \label{fig:flowchart}
\end{figure}

\newpage

\section{各機能・リソース詳細}

\subsection{1. 一覧機能 (Index)}
\begin{itemize}
    \item \textbf{URL}: \texttt{GET /prefectures}
    \item \textbf{処理}: サーバー変数の全データをEJSに渡し,\texttt{forEach}文を用いてテーブル形式でレンダリングする.
    \item \textbf{要素}: 「新規登録ボタン」,各行ごとの「詳細リンク」.
\end{itemize}

\subsection{2. 新規作成機能 (New / Create)}
\begin{itemize}
    \item \textbf{フォーム (GET /prefectures/new)}:
        \begin{itemize}
            \item 都道府県名,県庁所在地,地方の入力フィールドを表示.
            \item 送信先は \texttt{POST /prefectures}.
        \end{itemize}
    \item \textbf{作成処理 (POST /prefectures)}:
        \begin{itemize}
            \item リクエストボディから値を取得.
            \item 新しいIDを採番し,サーバー変数(配列)にpushする.
            \item 処理完了後,一覧画面(\texttt{/prefectures})へリダイレクトする.
        \end{itemize}
\end{itemize}

\subsection{3. 詳細表示機能 (Show)}
\begin{itemize}
    \item \textbf{URL}: \texttt{GET /prefectures/:id}
    \item \textbf{処理}: URLパラメータのIDに基づき配列を検索(\texttt{find})し,対象データを表示する.
    \item \textbf{要素}: 「編集ボタン」,「削除ボタン」(formタグによるPOST/DELETE送信),「一覧に戻るボタン」.
\end{itemize}

\subsection{4. 編集・更新機能 (Edit / Update)}
\begin{itemize}
    \item \textbf{フォーム (GET /prefectures/:id/edit)}:
        \begin{itemize}
            \item 対象データを検索し,\texttt{value}属性に現在の値を埋め込んで表示する.
            \item 送信先は \texttt{PUT /prefectures/:id}.
        \end{itemize}
    \item \textbf{更新処理 (PUT /prefectures/:id)}:
        \begin{itemize}
            \item IDに基づき配列内の該当インデックスを特定.
            \item リクエストボディの値でプロパティを上書きする.
            \item 詳細画面(\texttt{/prefectures/:id})へリダイレクトする.
        \end{itemize}
\end{itemize}

\subsection{5. 削除機能 (Destroy)}
\begin{itemize}
    \item \textbf{URL}: \texttt{DELETE /prefectures/:id}
    \item \textbf{処理}: IDに基づき配列から要素を取り除く(\texttt{filter} 等を使用).
    \item \textbf{挙動}: 削除完了後,一覧画面(\texttt{/prefectures})へリダイレクトする.
\end{itemize}
